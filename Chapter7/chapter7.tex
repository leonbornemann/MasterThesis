\chapter{Conclusion and Future Work}
\label{chapter_conclusion}

% **************************** Define Graphics Path **************************
\ifpdf
    \graphicspath{{Chapter7/Figs/Raster/}{Chapter7/Figs/PDF/}{Chapter7/Figs/}}
\else
    \graphicspath{{Chapter7/Figs/Vector/}{Chapter7/Figs/}}
\fi

This thesis studied the mining of complex events in data streams and how these events can be used in order to build predictive models for the data stream. In particular, the focus was put on very specific complex events called episodes, which are complex event patterns that are formed by combining basic events using the sequence or the conjunction operator. For the purpose of prediction, two algorithms were developed and implemented to train predictive models for categorical events in streams: \textbf{P}redictive \textbf{E}pisode \textbf{R}ule \textbf{M}ining in \textbf{S}treams (PERMS) and \textbf{F}eature \textbf{B}ased \textbf{S}tream \textbf{W}indow \textbf{C}lassification (FBSWC). Both algorithms mine the categorical event stream for training data and subsequently mine frequent serial and parallel episodes from them. PERMS then constructs an ensemble of predictive episode rules to predict the target event, while FBSWC trains a feature based classifier using the occurrences of episodes as features. \\
The evaluation on financial data streams was able to show that both algorithms can produce useful predictive models, which interestingly predict most of the extreme events (large increases or decreases in stock value) correctly. On average however, the predictive performance of the produced models is largely indistinguishable from random guessing and worse than a simple moving average, which means that, at least for the domain of financial data streams, the methods still need to be improved. However, the fact that many configurations show outliers with accuracy greater than 65\% shows that the algorithms are in fact able to build well performing models. The evaluation also showed that the addition of semantic knowledge can improve model accuracy in some cases but significantly increases the training time, since there are more patterns that need to be considered.\\
There are many future work opportunities in this area. First of all, it would be interesting to do an in-depth analysis of why the models produced by PERMS and FBSWC correctly predict many of the extreme increases or decreases in stock value. Finding out whether this is due to chance or whether the models actually found rules that allowed them to correctly predict the extreme cases would be very valuable information for refining the algorithms. Secondly, PERMS and FBSWC should be evaluated with data from other domains to see if the weak average performance in terms of accuracy is domain specific. Additionally the inclusion of semantic knowledge could be extended and more extensively evaluated as the evaluation has shown that additional events that are derived from semantic knowledge, can improve the predictive performance of the trained models. Furthermore, the suggested modifications to the algorithms as suggested in section \ref{sec_EvolvingModels} could be implemented and the effect of evolving models as the stream progresses could be studied in more detail. The predictive performance of evolving models could then be compared to the static case. While episode mining in streams has received attention, there are still many open problems, such as mining general episode patterns in data streams. Solutions to these open problems could contribute to an even larger set of possible patterns that can be mined, increasing the chance to find helpful correlations or rules that help predict events in categorical streams. Apart from that both PERMS and FBSWC could be adapted to work with different kinds of complex event patterns, which might in turn positively impact accuracy.