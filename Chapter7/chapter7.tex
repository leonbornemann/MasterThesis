\chapter{Conclusion and Future Work}
\label{chapter_conclusion}

% **************************** Define Graphics Path **************************
\ifpdf
    \graphicspath{{Chapter7/Figs/Raster/}{Chapter7/Figs/PDF/}{Chapter7/Figs/}}
\else
    \graphicspath{{Chapter7/Figs/Vector/}{Chapter7/Figs/}}
\fi

This thesis studied the problem of prediction in categorical event streams using episode patterns. Two algorithms were developed and implemented to train predictive models for categorical events in streams: \textbf{P}redictive \textbf{E}pisode \textbf{R}ule \textbf{M}ining in \textbf{S}treams (PERMS) and \textbf{F}eature \textbf{B}ased \textbf{S}tream \textbf{W}indow \textbf{C}lassification (FBSWC). Both algorithms mine the categorical event stream for training data and subsequently mine frequent episodes from them. PERMS then constructs an ensemble of predictive episode rules to predict the target event, while FBSWC trains a feature based classifiers using the occurrences of episodes as features. \\
The evaluation on financial data streams was able to show that both algorithms can produce useful predictive models, which interestingly predict most of the extreme events (large increases or decreases in stock value) correctly. On average however, the predictive performance of the produced models is largely indistinguishable from random guessing and worse than a simple moving average, which means that, at least for the domain of financial data streams, the methods still need to be improved. However, the fact that many configurations show outliers with accuracy greater than 65\% shows that the algorithms are in fact able to build well performing models.\\
There are many future work opportunities in this area. First of all, PERMS and FBSWC should be evaluated with data from other domains to see if the weak average performance as shown in the evaluation is indeed domain specific. Additionally the inclusion of semantic knowledge should be extended and more extensively evaluated as the evaluation has shown that additional events that are derived from semantic knowledge, can improve the predictive performance of the trained models. Furthermore, the suggested modifications to the algorithms as suggested in section \ref{sec_EvolvingModels} could be implemented and the effect of evolving models as the stream progresses could be studied and the predictive performance of evolving models could be compared to the static case. Finally, while episode mining in streams has received attention, there are still many open problems, such as mining general episode patterns in data streams.